\documentclass[12pt]{article}
\usepackage[margin=1in]{geometry}
\geometry{letterpaper}     
\usepackage[hyphens]{url}
\usepackage{fancyhdr}
\usepackage{enumitem}

\usepackage[parfill]{parskip}    % Activate to begin paragraphs with an empty line rather than an indent
\usepackage{graphicx}
\usepackage{amssymb}
\usepackage{epstopdf}
\usepackage{url}
\DeclareGraphicsRule{.tif}{png}{.png}{`convert #1 `dirname #1`/`basename #1 .tif`.png}

\title{$B^3$: Billions for Big Brains}
\date{}
\begin{document}
\vspace{-0.5cm}
\maketitle
\vspace{-2cm}
\begin{center}
Design a large scale proposal for understanding brain connectivity
\end{center}


\section{Request for Proposal}
Neuroscience is in a golden age of data and computation. With data acquisition systems being developed across
a broad range of scales, computational frameworks enabling scalable analysis unlike ever before, and growing
statistical and biological models enabling more realistic simulation of living organisms than we've seen to date.
We, Grelliam Corp., are formally issuing a request for proposals (RFP) in which we will fund a single group in the
amount of \$ 10~billion to uncover a mystery about the brain. Proposals responding to this RFP shall address a
challenge of the authors' choosing, spending up-to but not exceeding the available \$ 10~billion fund. Evaluation
will be based upon the considerations enumerated below, in Section~\ref{sec:eval}. Each team, along with the
submission of a proposal, will give a presentation to the review board which will highlight the key points of
their proposal.

\section{Background}
This project is inspired by a course entitled \textit{3YP} taught by Frank Wood at The University of Oxford.
The link to Frank's course is here: \url{http://www.robots.ox.ac.uk/~fwood/teaching/3YP_2016/}. On Frank's
website there exists a large number of resources, links, papers, and other tools that may be helpful in this
design process. It is important to note that the goal of the projects is different, so though you may benefit
from some of the resources contained within his site the objective, topic, and evaluation criterion stated
in this document should be followed in any discrepancy.

\section{Evaluation}
\label{sec:eval}
The proposals shall address several key considerations, enumerated and valued below.
\begin{itemize}[noitemsep]
\item \textbf{scientific question} (20 points) Each team must propose a scientific question that is of interest to solving
an important neuroscience challenge relevant to human health, computation, or technology,
\item \textbf{hardware + facilities} (10 points) What type of scanning/imaging equipment will be used to acquire images
of the brain is imperative. The scale of data collection may be up to the proposing team, and different modalities
may have significant advantages for specific scientific questions. Points will be awarded for describing the data
collection procedure in adequate detail such that the reviewers understand the efficacy of the chosen method in answering
the proposed question.
\item \textbf{data collection processes} (10 points) In addition to specifying the modality and necessary equipment to acquire
the images, considerations must be made to for how to get the images from the acquisition hardware into a format that enables
processing.
\item \textbf{information extraction processes} (20 points) Once data is acquired it must be processed and analyzed such that
the proposed scientific question can begin to be answered. This includes image processing such as registration, segmentation,
annotation, graph extraction, as a start. Additionally, once the given derivatives have been extracted, statistical analyses
must be performed to verify the findings and provide an indication of the significance of their strength. 
\item \textbf{data upload} (10 points) All raw and processed data, as well as their summary figures, must be uploaded to a
publicly available resource which enables other researchers and members of society to benefit from their existence.
\item \textbf{data storage} (10 points) Data storage must be considered for larger volumes, both in the short term when data
is collected from a microscrope and the longer term when it is being shared publicly for an indefinite period of time.
\item \textbf{cost} (20 points) A cost breakdown will summarize the expense of this proposal in as great detail as is possible.
Additional points will not be awarded if a team submits a project below budget (i.e. if the money is not spent it is considered
lost).
\item \textbf{feasibility} (50 points) This proposal must be feasible in the eyes of the review board, based on the details
contained within. Providing sufficient detail in order for this review to be possible is essential, as unanswered questions by
the reviewers will be considered holes in the feasibility of the proposal.
\item \textbf{total} (150 points)
\end{itemize}

\section{Submission Requirements}
\begin{itemize}[noitemsep]
	\item 3-5 page description of proposal in detail written in \LaTeX, thus in PDF format
	\item 3 slide presentation summarizing the proposal in PDF format
	\item 5-minute elevator pitch of the aforementioned slides.
\end{itemize}
 
\section{Submission Deadline}
The proposals must be submitted to Grelliam Corp. via a link to a Github repository by no later than 14h45 on Monday,
January 16th, 2017.

\end{document}  
